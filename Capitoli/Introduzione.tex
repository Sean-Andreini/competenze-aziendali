\documentclass[12pt]{article}

% --- Pacchetti base ---
\usepackage[utf8]{inputenc}
\usepackage[T1]{fontenc}
\usepackage[italian]{babel}
\usepackage{lmodern}      % font moderni per evitare errori
\usepackage{setspace}
\usepackage{geometry}
\geometry{a4paper, margin=2.5cm}
\setstretch{1.2}

\title{Competenze Aziendali}
\author{Manuele Pipaj}
\date{}

\begin{document}

\maketitle

\section{Innovazione e imprenditorialità}
L’innovazione e l’imprenditorialità giocano un ruolo decisivo nello sviluppo economico.  
Innovare significa introdurre qualcosa di nuovo, che può essere percepito dai sensi oppure riguardare un metodo di lavoro, un processo industriale, un servizio o un modello di business.  

Le innovazioni possono essere di tipo \emph{radicale}, quando esplorano nuove tecnologie e trasformano profondamente i mercati esistenti generandone di nuovi (come accadde, ad esempio, con l’introduzione dell’elettricità nelle case). Queste innovazioni hanno spesso risultati economici incerti.  
Al contrario, le innovazioni \emph{incrementali} utilizzano tecnologie già esistenti e si concentrano sul miglioramento di prodotti e servizi, riducendo i costi e aumentando la competitività. In questo caso il rischio economico è ridotto.

\medskip

L’innovazione non è semplicemente avere una buona idea: è un processo che si articola in varie fasi, dalla progettazione e dal test alla realizzazione fisica e infine alla commercializzazione.  
Affinché un’innovazione abbia successo è necessaria una corretta diffusione: informare i destinatari, formarli e convincerli ad adottare il cambiamento. La diffusione è sia un processo tecnico (marketing) sia un processo sociale, basato sulla comunicazione personale e sui media.

Il percorso di adozione di un’innovazione, a livello individuale, segue solitamente alcune tappe: inizialmente si ha una semplice esposizione (consapevolezza), poi cresce l’interesse, segue una fase di valutazione dell’utilità, quindi la prova pratica e infine l’adozione vera e propria.  

\subsection*{Profili degli adottatori}
Le persone che adottano un’innovazione non sono tutte uguali. Si distinguono cinque gruppi principali:
\begin{itemize}
    \item \textbf{Innovatori}, caratterizzati da istruzione elevata, orientamento al rischio e capacità tecniche specifiche.
    \item \textbf{Anticipatori (early adopters)}, persone autorevoli, con esperienze di successo e in grado di esercitare leadership sociale.
    \item \textbf{Maggioranza anticipatrice (early majority)}, composta da individui che riflettono attentamente prima di adottare un’innovazione, spesso con un forte legame con i propri pari.
    \item \textbf{Maggioranza ritardataria (late majority)}, più scettica e tradizionalista, con risorse economiche limitate e influenzata dalla pressione sociale.
    \item \textbf{Ritardatari (laggards)}, individui isolati, sospettosi, con scarse risorse e tempi decisionali lenti.
\end{itemize}

In sintesi, l’innovazione è un fattore determinante del cambiamento industriale: porta benefici temporanei finché le altre imprese non si adattano, ed è più diffusa nelle aziende giovani che in quelle consolidate.

\section{Rivoluzioni industriali}
Le rivoluzioni industriali rappresentano esempi di innovazioni radicali che hanno trasformato in profondità il sistema economico, produttivo e sociale.  
Nella storia recente possiamo distinguere quattro momenti fondamentali:

\begin{itemize}
    \item \textbf{Prima rivoluzione industriale} (1784): caratterizzata dalla produzione meccanica, dallo sviluppo delle ferrovie e dall’uso della macchina a vapore.
    \item \textbf{Seconda rivoluzione industriale} (1870): segnata dalla produzione di massa, dall’elettricità e dall’introduzione della catena di montaggio.
    \item \textbf{Terza rivoluzione industriale} (1969): contraddistinta dall’automazione, dall’elettronica e dalla diffusione dei computer.
    \item \textbf{Quarta rivoluzione industriale} (oggi): caratterizzata da intelligenza artificiale, big data e robotica.
\end{itemize}

Ogni rivoluzione industriale ha comportato un cambiamento profondo, non solo nei processi produttivi, ma nell’intero assetto economico e sociale.

\end{document}
