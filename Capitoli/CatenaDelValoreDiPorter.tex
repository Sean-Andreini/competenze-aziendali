\documentclass[a4paper,12pt]{article}
\usepackage[utf8]{inputenc} 
\usepackage[italian]{babel}
\usepackage{enumitem}

\title{La Catena del Valore di Porter}
\author{}
\date{}

\begin{document}

\maketitle

\section{Introduzione}
Michael Eugene Porter è un economista che ha definito la \textbf{catena del valore}, che prende il suo nome.  
Questo modello è particolarmente adatto per le aziende di prodotto di medie e grandi dimensioni, poiché consente la rappresentazione delle attività necessarie alla creazione del valore.

La catena del valore è un diagramma in cui vengono distinte due aree principali:
\begin{itemize}
    \item le \textbf{attività primarie};
    \item le \textbf{attività di supporto}.
\end{itemize}

\section{Attività Primarie}
Le attività primarie contribuiscono alla creazione fisica del prodotto o del servizio, alla sua vendita, al trasferimento al compratore e alle attività post-vendita.  
Le cinque fasi principali sono:
\begin{enumerate}
    \item Logistica in entrata
    \item Operazioni
    \item Logistica in uscita
    \item Marketing e vendite
    \item Servizi post-vendita
\end{enumerate}

\subsection{Logistica in entrata}
Attività associata con la ricezione, l’immagazzinamento e la distribuzione delle materie prime.  
Presuppone:
\begin{itemize}
    \item Scelta delle località di immagazzinamento;
    \item Layout e progettazione dei magazzini.
\end{itemize}

\subsection{Operazioni}
Effettua la trasformazione degli input nella forma finale del prodotto o servizio.  
Aspetti fondamentali:
\begin{itemize}
    \item Impianti efficienti e utilizzo della tecnologia necessaria;
    \item Layout ottimizzato;
    \item Attenta progettazione delle sequenze più adatte.
\end{itemize}

\subsection{Logistica in uscita}
Si occupa di raggruppare, immagazzinare e distribuire il prodotto ai compratori.  
Elementi chiave:
\begin{itemize}
    \item Processi di spedizione efficaci per garantire consegne veloci;
    \item Minimizzazione degli inconvenienti di trasporto;
    \item Spedizioni in grandi lotti per ridurre i costi.
\end{itemize}

\subsection{Marketing e vendite}
Attività legata alla definizione delle modalità di vendita e alla promozione dei prodotti o servizi.  
Aspetti essenziali:
\begin{itemize}
    \item Invogliare i compratori all’acquisto;
    \item Definizione esatta del segmento di clientela e dei suoi bisogni.
\end{itemize}

\subsection{Servizi post-vendita}
Fornisce i servizi volti a mantenere o incrementare il valore del prodotto.  
Comprende:
\begin{itemize}
    \item Risposte sollecite alle richieste dei clienti e alle emergenze;
    \item Qualità del servizio offerto;
    \item Formazione del personale;
    \item Monitoraggio degli eventi.
\end{itemize}

\section{Attività di Supporto}
Le attività di supporto aggiungono valore autonomamente o tramite la relazione con le attività primarie.  
Sono cinque:
\begin{enumerate}
    \item Acquisti
    \item Gestione delle risorse umane
    \item Ricerca e sviluppo
    \item Amministrazione
    \item Infrastruttura IT
\end{enumerate}

\subsection{Acquisti}
Si occupa di procurare gli input utilizzati nella catena del valore.  
Attività principali:
\begin{itemize}
    \item Acquisto delle materie prime;
    \item Sviluppo di relazioni \emph{win-win} con i fornitori;
    \item Analisi e selezione di fonti alternative per ridurre la dipendenza da pochi fornitori.
\end{itemize}

\subsection{Gestione delle risorse umane}
Attività di selezione, assunzione, addestramento e sviluppo delle risorse umane.  
Include:
\begin{itemize}
    \item Selezione efficiente e mantenimento delle risorse presenti;
    \item Qualità delle relazioni sindacali;
    \item Premi e incentivi per motivare il personale.
\end{itemize}

\subsection{Ricerca e sviluppo}
Riguarda un ampio insieme di attività che spaziano dallo sviluppo del prodotto al processo produttivo.  
Aspetti fondamentali:
\begin{itemize}
    \item Attività di R\&D efficaci per lo sviluppo di processo e prodotto;
    \item Mantenimento delle relazioni positive con altri reparti.
\end{itemize}

\subsection{Amministrazione}
Supporta l’intera catena del valore e non singole attività.  
Elementi chiave:
\begin{itemize}
    \item Sistema di pianificazione efficace;
    \item Ottime relazioni con i diversi stakeholders.
\end{itemize}

\subsection{Infrastruttura IT}
Fornisce il supporto tecnologico e informatico a tutte le funzioni aziendali.  
Aspetti rilevanti:
\begin{itemize}
    \item Gestione dei sistemi informativi;
    \item Supporto digitale ai processi produttivi e decisionali.
\end{itemize}

\end{document}
